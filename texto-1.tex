\documentclass{article} % Tipo de trabajo que se realizara
\usepackage[utf 8]{inputenc} %paquete de lenguaje
\usepackage[latin1, latin9]{inputenc} %paquete de lenguaje
\usepackage[T1]{fontenc} %paquete de lenguaje
\usepackage[spanish, activeacute]{babel} %Paquete de lenguaje
\usepackage{graphicx} %paquete necesario si quieres insertar imagenes
\usepackage{graphics} %paquete requerido si quieres hacer gráficas
\usepackage{multicol} %paquete de formato
\usepackage{tikz} %los paquetes tikz sera usados para crear graficas usando codigo
\usepackage{pgfplots}
\usepackage{tkz-fct}
\usepackage{xcolor}
\usepackage[usenames]{color}
\usepackage{amsmath} %paquetes necesarios para la compilación de matemáticas
\usepackage{amsfonts}
\usepackage{dsfont}
\usepackage{booktabs}
\usepackage{amssymb}
\usepackage{amsthm}
\usepackage{amscd}
\usepackage{amsbsy}
\usepackage{lmodern}
\usepackage{mathtools}
\usepackage{latexsym}
\usepackage{minted}
\usepackage{pgfplots}
\usepackage[papersize={216mm,271mm},tmargin=20mm,bmargin=20mm,
lmargin=20mm,rmargin=20mm]{geometry}
\usepackage{float}
\usepackage[none]{hyphenat}
\usetikzlibrary{calc, datavisualization}
\tikzset{every picture/.style={line width=0.2mm}}


\pgfplotsset{compat = 1.8}
\tikzset{flechaizq/.style={<-,>=latex}}
\tikzset{flechader/.style={->,>=latex}}
\tikzset{flechadoble/.style={<->,>=latex}}
\tikzset{punteada/.style={line width = 2, dash pattern = on 6pt off 3pt}}
\tikzset{continua/.style={line width = 1, flechadoble}}
\tikzset{conector/.style={line width = 2, flechadoble, dashed}}

\usetikzlibrary{shapes}
\usetikzlibrary{positioning}
\input{tcilatex}

%%Todas las lineas de codigo anteriores
%%no deben de ser modificadas por ahora

%% En esta parte escribiran el titulo del trabajo a entregar,
%su nombre o los nombres de sus compañeros (si el trabajo se hace en equipo)
%y la fecha de entrega.
\title{Primer tarea de Cálculo y de probabilidad usando \LaTeX}
\author{Alumno}
\date{ }

\begin{document}
\sloppy
\pagestyle{empty}
\maketitle
\thispagestyle{empty}
%%%%%%%%%%%%%%%%%%%%%%%%%%%%%%%%%%%%%%%%%%%%%%%%%%%%%%%%%%%%%
\section*{Problema de probabilidad.- De una caja que contiene $3$ bolas rojas,
$2$ blancas y $4$ azules, se extrae una bola al azar}  

\begin{enumerate}
    \item Encuentre la probabilidad de que sea roja
    \item Encuentre la probabilidad de que no sea roja
    \item Encuentre la probabilidad de que sea blanca
    \item Encuentre la probabilidad de que sea roja o azul
\end{enumerate}

\noindent Para calcular las probabilidades del problema solicitado
primeramente definimos el espacio muestral $S$ el cuál es:

\bigskip

\begin{equation*}
    S = \left\{roja, roja, roja, blanca, blanca, azul, azul, azul, azul\right\}
\end{equation*}

\bigskip

\noindent Definir los eventos a los cuales se les calcularan las probabilidades

\begin{itemize}
    \item $\left\{ X=\text{la bola sea roja}\right\} $

    \item $\left\{ X=\text{la bola \textbf{NO} sea roja}\right\} $

    \item $\left\{ X=\text{la bola sea blanca}\right\} $

    \item $\left\{ X=\text{la bola sea roja o azul}\right\} $
\end{itemize}

\noindent Se calculan las probabilidades de los eventos usando la regla de probabilidad simple

\begin{equation*}
    P(X)= \frac{\text{Número de casos favorables de }X}{\text{Número de casos totales en }S}
\end{equation*}

\section*{Respuestas}

% \begin{enumerate}
%     \item
%     \item
% \end{enumerate}

\section*{Problema de Calculo.- Graficar y escribir el dominio,
el contradominio y la regla de correspondencia de las
siguientes funciones:}

\begin{enumerate}
    \item La función lineal
    \item La función cuadrática
    \item La función exponencial de base Euler
    \item La función logaritmo natural
    \item La función Seno
    \item La función Coseno
\end{enumerate}

\section*{Respuestas}

\begin{enumerate}
    \item El dominio de una función lineal es el conjunto de todos
    los valores que la variable independiente (generalmente \( x \))
    puede tomar. Las funciones lineales tienen la forma general de regla de correspondencia:

\[
f(x) = mx + b
\]

Donde:
- \( m \) es la pendiente
- \( b \) es la ordenada al origen

Para este tipo de funciones, no hay restricciones en los valores
que puede tomar \( x \), ya que no se presentan denominadores que
puedan ser cero ni raíces cuadradas de números negativos, ni otras
operaciones que limiten el dominio.

Por lo tanto, el \textbf{dominio de una función lineal} es el
conjunto de todos los números reales:

\[
\text{Dominio} = (-\infty, \infty)
\]
\begin{tikzpicture}
    % Definir los ejes
    \draw[->] (-3,0) -- (3,0) node[right] {$x$}; % Eje x
    \draw[->] (0,-2) -- (0,5) node[above] {$y$}; % Eje y

    % Etiquetas
    \draw (-0.2,1) node[left] {1} -- (0.2,1);
    \draw (-1,0) node[below] {-1} -- (-1,0.2);
    \draw (1,0) node[below] {1} -- (1,0.2);

    % Graficar la función y = 2x + 1
    \draw[domain=-2:2,smooth,variable=\x,blue,thick]
    plot ({\x},{2*\x + 1}) node[right] {$f(x) = 2x + 1$};

    % Marcar la intersección en el eje y
    \filldraw[blue] (0,1) circle (2pt);

    % Marcar la pendiente
    \draw[red,thick] (0,1) -- (1,3);
    \draw[red] (0.5,2) node[above] {Pendiente $m=2$};

    % Etiqueta de la función
    \draw (2,4.5) node[right] {Función lineal};

\end{tikzpicture}

    \item
\end{enumerate}

%%%%%%%%%%%%%%%%%%%%%%%%%%%%%%%%%%%%%%%%%%%%%%%%%%%%%%%%%%%%%%%%%%%%%%%%%%
\newpage
\section*{Código fuente}
\inputminted{latex}{main.tex}


\end{document}